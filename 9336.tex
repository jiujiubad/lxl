\href{/web/}{\includegraphics{/static/images/toolbar/wayback-toolbar-logo.png}}

\href{/web/20130619202105/http://wordpress.lixiaolai.com/archives/9336.html}{JUN}

OCT

Nov

\href{/web/20130619202105/http://wordpress.lixiaolai.com/archives/9336.html}{\includegraphics{/static/images/toolbar/wm_tb_prv_on.png}}

11

\begin{figure}[htbp]
\centering
\includegraphics{/static/images/toolbar/wm_tb_nxt_off.png}
\caption{Next capture}
\end{figure}

2012

2013

2014

\href{/web/20131011170823*/http://wordpress.lixiaolai.com/archives/9336.html}{2
captures}

19 Jun 13 - 11 Oct 13

\href{}{}

\begin{figure}[htbp]
\centering
\includegraphics{/web/jsp/graph.jsp?graphdata=475_27_1996:-1:000000000000_1997:-1:000000000000_1998:-1:000000000000_1999:-1:000000000000_2000:-1:000000000000_2001:-1:000000000000_2002:-1:000000000000_2003:-1:000000000000_2004:-1:000000000000_2005:-1:000000000000_2006:-1:000000000000_2007:-1:000000000000_2008:-1:000000000000_2009:-1:000000000000_2010:-1:000000000000_2011:-1:000000000000_2012:-1:000000000000_2013:9:000001000100_2014:-1:000000000000}
\caption{sparklines}
\end{figure}

\hyperref[close]{Close} \href{http://faq.web.archive.org/}{Help}

\href{/web/20131011170823/http://wordpress.lixiaolai.com/}{Reborn}

\href{/web/20131011170823/https://twitter.com/xiaolai}{\includegraphics{/web/20131011170823im_/http://wordpress.lixiaolai.com/wp-content/themes/thesis_182/custom/images/twitter.png}}
\href{mailto:lixiaolai@gmail.com}{\includegraphics{/web/20131011170823im_/http://wordpress.lixiaolai.com/wp-content/themes/thesis_182/custom/images/gmail.png}}
\href{/web/20131011170823/http://wordpress.lixiaolai.com/feed}{\includegraphics{/web/20131011170823im_/http://wordpress.lixiaolai.com/wp-content/themes/thesis_182/custom/images/rss.png}}

\begin{itemize}[<+->]
\item
  \href{/web/20131011170823/http://www.lixiaolai.com/}{Home}
\item
  \href{/web/20131011170823/http://wordpress.lixiaolai.com/archives}{Archives}
\item
  \href{/web/20131011170823/http://wordpress.lixiaolai.com/downloads}{Downloads}
\item
  \href{/web/20131011170823/http://www.eduise.com/}{eduise.com}
\item
  \href{/web/20131011170823/http://www.eduise.com/client/evaluation}{免费留学申请选校评估}
\item
  \href{/web/20131011170823/http://www.lixiaolai.com/linode-vps}{linode}
\end{itemize}

\section{小人}

by xiaolai on 2010/05/11 ·
\href{/web/20131011170823/http://wordpress.lixiaolai.com/archives/9336.html\#comments}{93
comments}

in
\href{/web/20131011170823/http://wordpress.lixiaolai.com/archives/category/opinions}{言论与八卦}

韩寒被《时代》周刊评选为一百位影响世界的人物──这显然是很多\textbf{小人}不愿意看到的事情。对,不服气的都是小人\ldots{}\ldots{}不服气也没用,说的就是你!

``\emph{韩寒还年轻\ldots{}\ldots{}}''闭嘴吧,白活了一大把岁数的家伙──你们十年前就该闭嘴了。懂事儿之后,智慧和年龄没有直接的、必然的同比增长联系。说这话,露怯了吧?没意思了吧?

``\emph{警惕韩寒\ldots{}\ldots{}}''你自己爱警惕谁就警惕谁,谁都管不着;你爱怎么说就怎么说,法不管你就没人管你\ldots{}\ldots{}不过,要是你觉得你这么说,别人就跟着你一起警惕,露怯了吧?没意思了吧?

另外一个可敬的人落选了,韩寒入选了。可这根本就是两件事情,有些联系,却也不能硬拉在一起。这不是民众造成的,是什么造成的,很显然,大家都懂。``\emph{韩寒受推崇是庸众的胜利}'',此话一出,露怯了吧?没意思了吧?

不牛屄不怪你,装屄就没劲了。你是傻屄,也没人歧视你,但原本别人还不知道你是怎么回事儿的时候,你跳出来装屄,还被人看出来,那你不仅是傻屄,还是\textbf{自证的傻屄}\ldots{}\ldots{}呃呃,你先别急着反驳,用你的小脑仁想想,别露怯,别弄没意思了\ldots{}\ldots{}

你就这样老去吧。拜拜。

没人警惕你,你就习惯了这份孤单吧。

庸众从来都是胜利者,或者起码是受惠者,不信?读读历史书吧。

说完了,我很痛快。

天下

韩寒入不入选都一样牛B!\\ 他的博客写的犀利,读的痛快,不过被删的也快-\_-!

icu

兴,百姓苦;亡,百姓苦。多读读历史书吧。\\
百无一用是书生,在互联网上发个牢骚,一点用也没有,缺少的是董存瑞,黄继光。\\
韩寒无非是个开车的司机,某些人想让他做他根本做不了的事情,怎么自己不去当董存瑞,黄继光,撺掇别人,居心很良啊。。。

icu

引用一下,送给某些人吧。\\ ``露怯了吧?没意思了吧?''

icu

再说点没用的,人这一生就那么回事,都是瞎折腾。\\
无非是独立忽悠,组织忽悠,和被忽悠。所谓独立思考,纯忽悠,说白了就是自以为是。

\begin{itemize}[<+->]
\item
  caos

  你个SB。那你说说你对董存瑞、黄继光又了解多少!
\end{itemize}

ww

李老师怎么看待袁腾飞批毛?

\begin{itemize}[<+->]
\item
  joiho

  您老真闲
\end{itemize}

http://www.douban.com/people/2436009/ bai\_qi

蛋腚啊\\
评上了就评上了,是《时代》周刊作为一个组织对他的肯定。别人的一个意见,何必如此较真。另一些人对别人对某个人的意见的反对,又何必如此较真。哈哈

natalia

看的爽死了,记得六四的时候跟李鹏谈话的学生也被李鹏说:``你还年轻啊\ldots{}\ldots{}''。。。不过说实在的,这些S13爷爷们确实能耐够大的,有时候看他们做的那事儿就觉得的。。。真憋啊,当自由被束缚的时候比饿肚子还难受!

gordon

李总理是有资格说这句话的,看问题看的浅啊。90年代中国经济的腾飞,李总理居功甚伟;应该这么说,他和朱总理一个推一个拉,好搭档。简单的说就是需求拉动,技术推动。朱总理在外贸上有所作为,李总理在科技上完成了外贸出口产业的更新换代,只是时代发展太快,一些人的功绩被时间所淹没,后人再也不知道他的消息了。

wayne

= =。。。。

笑来不是说了``懂事儿之后,智慧和年龄没有直接的、必然的同比增长联系。''话都错了,还分什么资格不资格。。。。

再真说到资格,李鹏我还真想到一个事,他还真是居功甚伟

caos

你懂不懂啊,李鹏是死硬的左派,还``一个推一个拉''呢。朱镕基的升迁很大程度上是邓小平用来钳制李鹏、姚依林等人的力量的。\\
另外,他们归根结底都只是棋子罢了,又有什么``功绩''可言?照你这么说,我们反而应该歌颂``康乾盛世'',让康熙这个独裁者真的再活500年了?!

gordon

傻孩子,连笑来都是,他不过是找个借口罢了,那个借口有没有用,根本不是他们所关心的,重读\textless{}\textgreater{}
,估计你们上课的时候都在睡觉。

``韩寒还年轻\ldots{}\ldots{}''\\ ``警惕韩寒\ldots{}\ldots{}''\\
``韩寒受推崇是庸众的胜利''\\
和傻子真是无话可说,\textless{}\textgreater{}的文章自己找找看看。

\begin{itemize}[<+->]
\item
  gordon

  里的文章是``狼和小羊''
\item
  gordon

  说笑来是个``傻孩子'',是因为有些事可不是``小人''这么简单,有些问题根本就不是``人性''或者''民族性''或者直白的说``中国人的劣根性''的问题,谁要是信这个,谁就是白痴。\\
  闲了没事,读读``联邦党人文集''吧,这本书回答了所有的问题。\\
  这个问题曾经让包括富兰克林、华盛顿在内的美国一众开国元勋束手无策。
\end{itemize}

gordon

感谢caos对我的指点,触发了我的思考,让我解决了另一个问题,让我对宪政有更深一层的了解。\\
那个问题困扰我差不多有十几年,就是冷战时期毛哥利用``无政府主义''对抗``民主制度''的问题,虽然你说的和我的问题没有关系,但还是谢谢。

natalia

where is my comments\ldots{}.

\begin{itemize}[<+->]
\item
  gordon

  没有调查就没有发言权,所有所谓的民主派普世派都应该去\\
  ``富士康''这家公司干一段时间,再来说自己的问题 ;
  实在不愿意去,找找以前的日本老电影《野麦岭》看看,和当今的``富士康''是多么的想象,连人说话的语气都一样,
  ``辛辛苦苦几十年,一夜回到解放前'',如果连这一关都跨不过去,就不要谈民主、宪政,无论\\
  从理论到实践都需要重新充电学习。
\end{itemize}

vernsu

麦田老师看完啥感觉。。。

delphi

挺欣赏他的,对他没意见。\\ 但不觉得韩寒有什么大不了的影响力\\
如果说当今世界找不出100个甚至1万个比他更有影响力的人物来,那肯定是个笑话

\begin{itemize}[<+->]
\item
  Bruce Lee

  影响有多种:萨达姆有萨达姆的影响;奥巴马有奥巴马的影响;钱学森有钱学森的影响
  etc. 不要否认韩寒的影响。
\end{itemize}

delphi

另外顺便说一下,我并不特别欣赏许知远,觉得他文字绉绉得很,不过他的那篇
《庸众的胜利》基本的点我都赞同------

``于是谈论韩寒,变成了一次全方位的心理按摩。你沐浴了青春、酷、成功、机智、还觉得自己参与了一场反抗,同时又是如此安全,你不需要付出任何智力上、道德上的代价,也没有任何精神上的仿徨,他是这个社会最美妙的消费品。''

kevin

个人感觉,许知远总有些自命清高和做作,相较之下,韩寒的文字幽默、平易近人还一针见血。\\
支持韩寒。

\begin{itemize}[<+->]
\item
  delphi

  不错,许知远总有些自命清高和做作,就是我说的``绉绉''得厉害,而且文章总差点力道,总有股想进入某地就是进不去的劲儿
  (不要想歪了)

  其实,我想许同学不喜欢的不是韩寒本人怎么样了------韩寒本人挺好的,除了私生活曾经一段时间稍稍乱了点同时和四五六七八个小妹妹交往,另外消费大手大脚无论个人生活和职业都不符合目前环保低碳的主题以外------人年轻,长的精神,赛车虽不环保但是酷,文章写的爽------本人没有问题,但是韩寒现象,这个phenomenon,确实很bizarre------说白点,就是有些人太当真了

  京城不少文化人,韩寒一出一个新博客------我琢磨着其实人家也就是赛车泡妞之余噼里啪啦半小时打出来的一篇文章------那些文化人就可以立马奔走呼号,纷纷在新浪围脖、个人博客中广而告之------韩寒又出新博客啦!快来看啊!然后还都真仿佛文化大革命引用毛选一样顺便引用其中的一段话------这个,是非常bizarre的
\end{itemize}

随便说说

呵呵,如果笑来这篇博文是针对许知远的话,我觉得笑来反应过度了,不知道笑来耐心读过许知远的这篇文章没有。

其实许知远文中批评的并不是韩寒,而是这个社会,或者说社会风气,社会现象。而且,许知远的这种批评立场是一以贯之的,并非是因为受到韩寒入选Times刺激,出于忌妒或别的不良情绪而针对韩寒而写,故作此姿态------至少以我这两年阅读的许的文章的经验来看是这样。

Nova

我上大学以前看过许知远的《那些忧伤的年轻人》,觉得里面写到的很多东西还是很不错的,讲的是理想化的大学教育,感兴趣的可以看一看,有电子版。可能由于他和韩寒的文字都是我比较喜欢的,所以《庸众的胜利》只是简单的看了下,还没去认真想到底谁错了。

十年前韩寒在对话上,被人说年轻的时候,他的反驳还显得挺无力的,现在还是有足够的能力反驳了。

《警惕韩寒现象》写的还是有一点水平的,至少是拿事实说话了,虽然说出的道理好像没那么有理。

李笑来

没道理且有水平\ldots{}\ldots{}我晕了。

\begin{itemize}[<+->]
\item
  Nova

  我之前说的随意了一点,没表达清除。\\
  我的意思是这样的,《警惕》一文中列出的许多现象都是正确的,反应出了作者的观察能力是有一定水平的。比如说作者指出\\
  ``1,韩寒博客的内容,在2008年后非常``锁定''在社会热点问题,特别是网络热点话题。\\
  2,韩寒博客的文字,语感极强,尤适网络阅读。\\
  3,韩寒博客的立场,几乎都占在公众认为的``弱势''一面。''

  我认为作者指出的这些都算事实,作者指明```热点`,一流的`文字`,正确的`立场`,这就是韩寒的魅力。''这的确说明作者有一定的水平,而且是认真观察和思考的。

  我所谓的``没道理''是指:由这些事实,推导出``要警惕韩寒``这一结论的论证是没道理的。作者的逻辑是韩寒的说法不是理性的,但都是老百姓``爱听''
  的,所以要警惕韩寒''。

  我觉得作者的一个主要的错误,是说韩寒的作品``不理性'',例如作者指出``韩寒的文章压根不去分析事实,而是绕口令一般继续传递不信任'',我觉得韩寒的文字大多比较理性的,比如说在《你们在怕什么》一文中强调了``这件事情我并不了解,在其他众多的维权事件中,政府一定全错么,不一定,维权者一定全对么,也不一定。但是为什么政府永远表现出全错的态势呢?''
  。表现出了韩寒对于不清除的事情还是比较理性的,这也是我喜欢他的一个原因。

  ----------------------------------------------------------------\\
  不知道笑来老师能否耐心的看一下,这样的表达是否清楚,谢谢!
\item
  Nova

  我之前随手打的,没有表述清楚,我的意思是这样的:麦田指出的关于韩寒博客的几个事实,在我看来是正确而且观察细致的,我觉得这让这篇文章有一定的水平。但是在这些有这些事实推理出要警惕韩寒这一论证过程,是没有道理的。

  关于麦田提到的事实,比如说:\\ ``哪里有热点,哪里就有韩寒!''\\
  ``韩寒博客的文字,语感极强,尤适网络阅读。同时,文字幽默、讽刺、善用暗喻等等。''\\
  ``韩寒博客的立场,几乎都占在公众认为的``弱势''一面。''\\
  我觉得都比较有道理,总结的很到位,是显示出麦田有一定的水平。

  但是,麦田由此推导出,``这两个说法,都不是理性的,但都是老百姓``爱听''的。读到这里,我才顿悟:要警惕韩寒'',是没有道理的。首先是因为''韩寒的观点不理性''这一前提可能并不成立,比如说韩寒在《你们在怕什么》中就指出``这件事情我并不了解,在其他众多的维权事件中,政府一定全错么,不一定,维权者一定全对么,也不一定。'',韩寒显得比许多人理性得多。所以批评韩寒不理性是麦田文章中最主要的一个问题。其次,就算韩寒故意的迎合百姓的情绪,也不置于就需要警惕。

  麦田引用到的另一个观点其实也挺不错的,就是``为老百姓说话''不等于``说老百姓爱听的话''。许多为百姓说话的经济学家(比如说茅于轼)说的话百姓的确不爱听。当然,我觉得韩寒显然是前者,麦田指责他是后者是缺乏依据的。

  总之,我觉得麦田的文章虽然主要观点是错的,论证显得无道理,但是文中是有一些正确的部分的。我个人评价``大负小正''吧。

  ----------------------------------------------------------\\
  不知道这样表达是否清楚了,谢谢笑来老师指出不足。
\end{itemize}

Kenny Yuan

事实说话?只是精心挑选的事实罢了。他的文章里最核心的``事实''其实是他选择性失忆了。比如抵制家乐福事件时,村姑的发言就是站在``多数人''的反面的

\begin{itemize}[<+->]
\item
  nova

  这一点我比较同意,他写了那么多博文,大多数应该来说都是比较理性的,但的确也可能会有一两篇不那么理性的,往往就会被攻击者挑出来。家乐福事件的时候他的确很理性,站在了大多数不理性人的反面。
\item
  Nova

  这一点我同意,家乐福的时候韩寒是理智的站在了大多数不理智人的对立面。韩寒写了那么多博文,难免是会有几篇出现一点不理性的情况。总的来说,他的大部分文章都是比较理性的,这也是吸引我的一个主要原因。
\end{itemize}

刘磊

看完了,我很痛快,因为我好像看懂了。

lyon

韩帅不过是在artists组,跟Lady Gaga一起。。。所以。。。不要这么认真

\begin{itemize}[<+->]
\item
  k

  作家不是artist是什么?看不起artist?
\end{itemize}

若微

上推的好处是\\ 这篇博客笑来想说谁谁谁,我都知道了

相忘于江湖

转twitter上一个人的评论。@bitinn《韩寒受推崇是庸众的胜利 》内涵有三:1.
作家许知远不是攻击韩寒,而是攻击中国的民族性无知。2.
作者认为大陆缺乏说真话的人是一个民族的失败。3.
本文最初刊登在亚洲周刊,新华网很聪明的把攻击点从人民转移到了韩寒上,你不得不说真是老奸巨猾
\href{/web/20131011170823/http://is.gd/c3A49}{http://is.gd/c3A49}

李笑来

新华网删除了原文第一句话:

\begin{quote}
韩寒说出一些聪明话,时代神经就震颤不已,这是庸众的胜利或民族的失败。
\end{quote}

相忘于江湖

我认为许知远的错误在于其实韩寒在这个社会上其实还不能算是主流,社会主流依然认为朝廷好,地方坏,韩寒的文章春风润雨般的指出这个根子是烂的,看他文章的人不会连这点都看不出来,为啥韩寒提不出所谓建设性的意见?因为任何一个建设性的意见都是对政权的颠覆。而小四的粉丝和收入远超韩寒,这才是庸众的胜利。

\begin{itemize}[<+->]
\item
  http://blog.sina.com.cn/fanfouer Benj

  没必要拿小四出来与韩寒对比。

  这个社会趋向多元化,多元价值;肥皂剧也有它的受众。也不能说受众是庸众。受众里是一个个独立个体,他们有不同职业、爱好、家庭、地域、追求;爱看小四只是生活的一小部分。

  别贴标签。我们都是活生生的人。

  可不可以这样说:当一个作者把自己与``庸众''分别,他就趋向于凌驾``庸众''之上,觉得自己能为他们树立标竿、能引领他们、觉得世间我更好。好像慢慢滑向了主宰者的边缘。

  我,个人,不要这个标签,自在的好,我有所坚持、有所向往。
\end{itemize}

gordon

牛叉的新华网,就改了一句话,整个文章意思大变;新华网还是有高手的。

从解构主义开始兴起到现在,我们的社会对应兴起的是重新看历史的思潮。实际,这样的思潮为统治者所利用,并借助结构主义,把细节拆分,把他们需要的情节无限的放大。于是,政治的正确与历史的正确构造出一个我们今天熟悉的世界观。

五毛就是这样被创造出来的。五毛为什么如此困惑,因为我们宣传口径向来就是这么宣传的。讲述这个问题,又找到了一个好例子。呵呵

准备写一本书《从日子到月子--五毛是怎么被创造出来的》

\begin{itemize}[<+->]
\item
  gordon

  更正,``这样的思潮为统治者所利用,并借助结构主义'',\\
  不是``结构主义'',是``解构主义''。
\end{itemize}

http://blog.sina.com.cn/fanfouer Benj

我的天啊。\\ 人性从来没有多少进化。趋利避害是一个。

所谓民族性、劣根性这些``名词''只不过是造成心理分裂而已,或是安慰。

外来的侵略加剧了这种分别心、以为我们跟别人不一样。\\
其实,我们都是人,是人,就都有人该有的特点,有些人对自己要求高些,就强化某些特点而弱化某些特点。

对人性的恶,在公共范围内要用制度去制衡。

仅此而已。庸众如我者就该干嘛干嘛。Do what I have to do ,SO I can do what
I want to do.

http://blog.sina.com.cn/fanfouer Benj

鲁迅批判民族性很厉害的,病死了;那些直接作战的人们被暗杀了、枪毙了--我都不知道他们的名字,历史书的功劳簿上都没有他们的名字;因为还没有民众写的历史书。历史是民众创造的,可历史书不是。

鲁迅作品里与许多人斗争,好像他更难过的是那些叛变的文化走狗,鲁迅想对准``直接敌人''--权势者;可是权势者不鸟他,只消收买一些文人就可消耗鲁迅的许多精力。作鲁迅很累的。因为他站在弱者的一边,很辛苦的;同盟少。\\
在恶势力面前,哪里没有阿Q!?

我看韩寒,也不妨碍认同刘阿波;当然认同也不需要嚷嚷出来。许知远对现状可能焦虑,个人的事;许西方书看多了,常拿同时代的中西方比较,难免焦虑。

需知,伟大的工程需要三百年。我们还需要时间``文艺复兴''--把扭曲人性从传统重压下解放出来;``启蒙思想''--权力解构与重建;技术革新--超越墙,促进信息流通;自我提升--卖体力脑力产品不卖人品。

你--庸众--我:你、我之间不需隔着一堵``庸众''的墙。我们的幸福随时都会不见,这就是为什么许发表在亚洲周刊的文章有敏感词--在权力被关进牢笼之前,你、我、许知远、韩寒都是庸众--\\
真相会引导我前进。\\ 不要说教,童鞋;\\ 尽你所能,给我多D真相。I Need More!

http://www.gooqor.cn 够抠

韩少还是那个韩少,从``少年啦飞驰''认识的\textasciitilde{}

========继续问,直到有人答===================\\
请教笑来老师和路过的大大们:

笑来老师推荐的柯林斯英语词典的纸质版本是哪一个?

市面上的种类何其多啊?什么高阶 高级 重复名称的也很多
各种出版社的,查不出眉目呢!笑来老师博客里的是电子版的,与电子版对应的有没纸质版的啊!?要能在国内买到的啊,别给咱说亚马逊美国\textasciitilde{}汗

谢谢来\textasciitilde{}

safelysun

笑来威武!!!

gordon

致五毛党

在学术体系已经欧美化的今天,本不想多说什么,但还有很大一部分人受的还是传统教育或者说曾经受过传统教育,我尽量把传统教育理一遍,把它讲的是什么说清楚,这并不是一件容易的事。

我们都知道``学术传统积累的重要性'',其实做学问最忌讳的就是``推倒重来''。苏俄的学术传统就是靠日积月累最终导致井喷式爆发的--

前面在讲到李开复博士的时候,我说李博士就是那个风格,而我们是另外一种风格,所有的风格都归属于一个体系,单独的开局是不行的。那么我们的风格是什么,我们的流派是什么?

这个事从什么地方说起呢?还是先听一首歌罢.

\href{/web/20131011170823/http://v.youku.com/v\_show/id\_XMzUzODY5MDQ=.html}{http://v.youku.com/v\_show/id\_XMzUzODY5MDQ=.html}

听完这首《葬花吟》,我们来说一个人。

\begin{itemize}[<+->]
\item
  gordon

  本来想说陈天华的,有点长就不写了。
\end{itemize}

您独立思考了吗?

真无语\ldots{}\ldots{}\\
人们一直在强调``独立思考'',可是到底有多少人真的做到``独立思考''了?为什么提到韩寒就马上有人跳出来说别人是小人?\\
说实话,我觉得麦田那篇文章和许志远那篇文章根本不能相提并论。对于前者,我也是笑笑而已;可对于后者,我不认为许先生是出于``小人之心''而写的文章。这里有篇文章我个人觉得分析的不错,希望冷静思考的人能看看:\\
\href{/web/20131011170823/http://www.fangkc.cn/2010/05/hanhan/}{http://www.fangkc.cn/2010/05/hanhan/}\\
作为庸众的一份子,有多少人真的是有所作为的?有所作为并不光是指去闹事,去推翻xxx,像韩寒利用他的影响力去开启民智,毫无疑问是作出了巨大贡献有所作为的,因此这篇文章根本不是批评韩寒。可是某些读者,您除了看韩寒文章的时候叫一声好,还做了什么?宪章签了吗?三网民您声援了吗?如果您做过哪怕一件实事儿,那这篇文章就不是说您;可如果您除了在屏幕面前叫好之外什么都没参与过,您不是庸众是什么?\\
韩寒牛逼,不等于韩寒的读者就牛逼了。同样,骂韩寒的(某部分)读者,也不代表在骂韩寒。在反骂的时候逻辑先搞清楚的好。

相忘于江湖

我同意你的观点。许知远在文章的结尾说道:``韩寒掀起的迷狂,衬托出这个崛起大国的内在苍白、可悲、浅薄------一个聪明的青年人、说出了一些真话,他就让这个时代的神经震颤不已。与其说这是韩寒的胜利,不如说是庸众的胜利,或是整个民族的失败。''就好比这里的读者读了笑来老师一系列的时间管理和英语学习指导文章后都高呼这他妈的太好了,看完后我的大脑就像重生了一样,有了笑来老师的英语学习方法指导,我相信我也能学会英语。但光有这样的欢呼有个屁用,这里看了笑来博客两年以上的人多得是吧,有多少个人在这两年里学会英语了?有几个人像笑来老师的把时间当做朋友中那样刻苦的做过事情?如果有个人对笑来老师说你说的太有道理,我也要拿下英语,但我没有时间学习,笑来老师您有什么高招吗?笑来老师对于这样的人有何感想?

wayne

我同意你说的``做实事儿'',但民众大体上说来总是无辜的,原因不就是因为这个吞噬人的巨大体制么。如此舍本逐末显得太奢侈啦。而且许知远自己躲在香港写这种文章在我看来也蛋疼的很。

您独立思考了吗?

至于大众无辜论,我想说的是,推特上相关讨论已经很多了:每一个公民,都有政治权利和政治权力,因此,政治生活,是每一个社会人最基本的生活,当今中国最大的灾难,就是绝大多数人丧失了政治热情,由于恐惧而放弃了自身拥有的政治权利和政治权力,这是政治之所以被极少数人垄断的根源,因此,我们几乎没有资格诅咒专制,因为我们每一个人对专制的形成负有不可推卸的责任。\\
至少我个人是认同该观点的,也许您的看法不同吧,这个暂且不论。但是,您说许知远``躲在香港''写文章很蛋疼。这我要怎么说才好?\\
首先,维基上说``(许知远)从2009年8月起到2010年8月,为时一年时间在英国剑桥大学交流学习一年。'',其次,FT中文网上有他早期的一篇文章,他自己说自己是基本上一直住在北京的(在去英国学习之前)。\\
其次,可能您平时可能比较少看他的文章,在我看来,他的文章尖锐程度绝不弱于韩寒,并且一定程度上更有深度(``有深度''不等于比``通俗易懂''更牛逼,这里需要注明,免得``某些''韩粉过于激动。)。况且即便光就这篇文章而言,也只有潜意识里误认为``该文的中心思想就是批判韩寒'',才会觉得他是蛋疼行为。\\
------------------\\
窃以为,公众人物的最大价值就是启发民众,是把独立思考的精神带给普通大众。如果韩寒看到他的粉丝都高叫``韩寒一定是对的!'',如果笑来看到他的粉丝都高喊``笑来一定是有理的!'',我想,如果他们有哪怕一点点良知,也应该对此现象感到悲哀\ldots{}\ldots{}

stupid

您有良知,请您继续悲哀。

\begin{itemize}[<+->]
\item
  您独立思考了吗?

  我那段话的意思是为了批评韩寒和笑来没良知么?没有看到我前面有个``如果''做假设么?我只不过是在强调一个看法:``独立思考是重要的,造神是不可取的''而已。\\
  在许知远文章这件事上,我从来没觉得韩寒错了,也没觉得许知远错了,实际上我早就强调了:许知远根本没有批评韩寒,他批评的是``只满足与看韩寒文章而无实际作为的大众''。(当然,这是我个人的思考和理解)
\end{itemize}

v

你明不明白啊?本來普通人,庸眾,就不應該做太過激動的事,誰應該去做啊?就像許知遠這些,像韓寒這些,為什麽?因為他們有名,這種名氣賦予他們更多安全感,因為他說一句,即使再傻逼的話,也有人聽。你看看除了我們之外的國家,明星,演員和作家那些擁有名氣的人都是有極度鮮明的立場,往往是某一個基金和組織的帶頭人,幫忙籌款,從來不害怕表達意見,即使是多么激烈。

我們沒有NGO,我們的名人集體失聲,竟然有人要求普通人再激烈一樣,瘋了?

許知遠基本上和韓寒一樣,(當然前者沒有後者那麼紅)他們不會做太激動的事,問題是許知遠自己不做的前提上,指責普通人''不夠危險``''太過安全``,屁,不是小人是什麽啊?

\begin{itemize}[<+->]
\item
  您独立思考了吗?

  您说的有一定道理,但我想再补充一点:\\
  假设我们把所有事情的危险度用10分制来区分,假设名人们做6分的事情还能保持安全,但做7分就有危险了;那可能对于普通人来说,做3分的事情就是最大安全度了。\\
  问题在于你怎么去理解许先生的``蛊惑''?您是认为他在蛊惑您做
  \textgreater{}3分的事情,还是在``蛊惑''那些连1分的事情都不愿去做的人行动起来?\\
  我倒是觉得,``聪明的对抗''是任何人都应该学会的。在这方面,韩寒是榜样,他绝对不越界,但又能恰到好处的启发大众,也就是说,他做到了他所能做的最大限度。在我看来许知远也一样。那对于普通大众,有多少人做到了自己的``最大限度''?好吧,也许不该要求这么高,可是现实是很多人不仅没做到3分,连1分都没做到。\\
  许先生只不过是在提醒行动的重要性罢了。很多人说,韩寒是中国的希望,是啊,这个希望就像是火种一样,可是如果大众只满足于对其喝彩,而不去行动,不去把这个火种变得更旺盛,何谈``会有千百个韩寒''??\\
  不要一听到``行动''就觉得是让你去游行示威,如果觉得只有这种``愚蠢的行动''才叫行动,那确实应该怀疑一下自己的理解能力。
\item
  v

  首先我們通過新聞瞭解,這個激動在不同的省份有不同的標準。\\
  想請問一下你,你住哪個省份?,全部省份你都住過嗎?你法律常識如何?你懂得如何做到在法律的範圍裏表達意見,并在不太懂法的熊貓找上門的時候用自己的知識為自己辯護嗎?好吧,如何你可以,如果你居住的身份足夠自由,你就去做吧,做1,3還是7分我都不管你。但你有什麽權利?你爲什麽對自己的智慧那麼有自信?確定你可以用你自己的10分制來衡量別人的過激標準?也許你覺得1分的事情可能很多人覺得是9分了,那怎樣?你有權利去指責別人不夠你勇敢嗎?你的十分標準,請你留著,誰讓你做上帝啦?別人覺得危不危險難道還要你決定啊?\\
  有很多的事情是自己來衡量自己,要求自己的。我承認我沒看過所有韓寒的文章,但這帥哥好像從來沒指責那些本來就手無寸鐵的人不夠勇敢,不夠激進;他就愛諷刺那些本來權利在手的人濫用權力之類的。對普通人,達到``不作惡''這條就免責了。\\
  是連嶽還是笑來曾經說過,我忘了,''要求所有人都有品的人也太沒品了。''這句話送給你,也送給那位許先生,你們勇敢你去做,做完效果好就會激發無數人,你放心,我們一直都在學習如何變得聰明,從學習真正勇敢聰明的人,揭露偽勇敢的人開始。\\
  還有,所有是追求光明的行動都不愚蠢。
\item
  您独立思考了吗?

  @v:\\
  回复的层数太多,似乎无法直接在您发言下面回复了,不知道写在这儿格式会不会乱掉。\\
  哎\ldots{}对于您偷换概念的做法,我也许应该理解为是您曲解了我的意思造成的,不过呢,也许是我表达水平的问题,谁知道呢?那我索性再罗嗦的解释一遍:\\
  1.``你的十分標準,請你留著,誰讓你做上帝啦?別人覺得危不危險難道還要你決定啊?''\\
  我从来没有想做上帝,我所说的10分标准也不过是形象化的描述如何去衡量自己的行为可行度。实际上,这个标准应该在每个人自己心里而不是个统一的标准。我倡导的其实很简单:做自己力所能及并且不危险的事。如果您觉得签宪章过于危险了,那可以上网发帖让更多人知道这些东西;如果您觉得发帖也太危险了,您可以平时向周围的人普及一下这方面的知识,让他们不要认为ccav就是真理\ldots{}\ldots{}我想,无论再小的行动,总有您力所能及并且不危险的吧?\\
  2.``有很多的事情是自己來衡量自己,要求自己的。''\\
  ``要求所有人都有品的人也太沒品了。''\\
  是的,您说的太对了,这两句话我绝对不反对。我觉得您对我的最大误解就是认为我在强制某人行动。其实真的不是的,这只是对整体的建议而已。我从来没说过``不行动的人都该去死'',也从来没表达过这种意思。我反复强调的是:如果``大众''只停留在``看韩寒博客''的水平上(对整个民主进程)是不够的,社会需要更多的韩寒,我也衷心希望有更多的韩寒行动起来。但这并不是针对某一个人说:我命令你必须行动。\\
  作为集体的一份子,每个人都有权利去反思去批评整个集体;但的确没有人有权利去批评某一个个体。就比如鲁迅批判中国人的劣根性,人们都觉得他说的对;可是假如他批判的是某一个具体的人,我想他就侵犯了这个人的某种权利吧。\\
  3.``所有是追求光明的行動都不愚蠢。''\\
  我觉得吧,非自然科学领域之内,绝对化的命题一般都是假命题。当然,也许您这句是``一般''之外的命题,另外,也许我还可以把它理解成是``带有修辞手法的句子'',而不是严格的结论。呵呵\ldots{}\ldots{}\\
  ------------------------------\\
  最后我想说的是:我觉得本来很简单的一个思想,没想到能有这么多误解,也许我的文字水平实在是狗屁不通吧,但是我已经试图解释到最清楚了,如果还会误解,我也没太多的时间来慢慢解释。我一不是``公公''知识分子,二不想做人民的``领袖'',之所以罗嗦这么多不过是很朴素的希望大家能更理智的看待许先生的文章而已。如果大家实在不能认同我的看法,那也再正常不过了。\\
  就此为止吧,衷心感谢和我讨论问题的各位。我不再回复了,祝大家愉快。
  \^{}-\^{}
\end{itemize}

Fans

围观自证的傻屄的新文章

Fans

为什么那么急于论证别人比自己更傻逼呢,亏您还妄称学习脑科学。

⑦號桌銶

那你是在论证什么?

Fans

您觉得我这是在论证吗?先搞清楚这点再说吧

⑦號桌銶

那您这是直接用反问的语气来下结论?您有资格么?

\begin{itemize}[<+->]
\item
  stupid

  别跟这个书名Fans的争\ldots{}\ldots{}他还以为在围观,其实是在被围观。
\end{itemize}

http://blog.sina.com.cn/jasonzzs Jason

笑来老师,我也支持韩寒,也写博反驳过《警惕韩寒》。但是我觉得你这篇文章写得忒没意思,对韩寒当选100位影响世界人物有异议就是不服气?别人就不能认为有其他100人比他更有影响力吗?就算是不服气,就应该打上``小人''的标签么?难道就不能允许别人对韩寒置疑吗?这种逻辑和乱扣帽子的做法与某党何异?

说回来《时代》的评选方法也就那么回事,最后也就是拼上网的粉丝数量,跟姚明当选全明星票王差不多道理

\begin{itemize}[<+->]
\item
  HM\_S

  fyi 韩寒的票可都是贴吧里那些粉丝辛辛苦苦刷出来的
\end{itemize}

火星人

爱韩寒,爱笑来。

zeugma

天下大事,那么多,一个杂志的评选,一个皇帝新装里的小男孩,值得那么较真吗?

美丽心情

很痛快!可否转载?

sattieryy

支持韩寒!有些人大可不必烦恼,韩寒至少证明了天朝还是有言论自由的,没读过大学的人照样可以活得风生水起。

secrpa

这不是民众造成的,是什么造成的捏?是警惕韩寒的人造成的呗。来无踪,去无影,我乃常山赵紫阳是也

la-la-land

韩寒嘛,小伙子长得倒还不错

\begin{itemize}[<+->]
\item
  david

  我对你很感兴趣
\end{itemize}

http://www.onlyswan.net Eason

仔细读了许知远文章,觉得还是很中肯的,韩寒很牛逼,韩寒很不错,可是这个社会只有韩寒,只有一个韩寒远远不够。

http://www.onlyswan.net Eason

我认可许知远的观点,同时也支持韩寒。\\
这就像,虽然我知道我的女朋友不是国色天香,但我仍然很爱她一个道理。

abigail\_dun

笑来老师说得好!有多少人能有韩寒的思维,中国还需要更多韩寒,而非商业化的郭敬明。\\
最近我看了张艺谋的巅峰之作------《或者》,这是一部被禁的电影,但足以被写入中国历史,现在中国更需要的是类似这样的片子,而非一味为四大发明、为孔子、为过去臭美,早该珍惜的时候不珍惜,看见自己的文化被别人抢了的时候才来作秀,不正视自己国家过去的的人和国家都不会有颠覆性的改变和突破。

deinonychus

许知远那篇文章貌似还写到一个人叫 LIUXIAOBO。

\begin{itemize}[<+->]
\item
  david

  对,不能忽视LIUXIAOBO的存在。
\end{itemize}

myspan

看了徐的文章,觉得他的中心在于说明韩寒的成功是因为大众的懦弱与浅薄,并非说韩寒有什么不好。类似的有袁腾飞,一个教高中的历史老师给大学生普及历史,流行的原因同样正是大众的无知。徐看到了一大批庸众,所以说出这番话,但是没有这两个人,连庸众都产生不了。只能说徐太心急了。\\
笑来反应过激了

\begin{itemize}[<+->]
\item
  zoe

  对,许知远文章的意思比较绕,没有韩寒的易懂、可读性强。\\
  我也认为他的意思主要是说韩寒的流行说明我们读者的失败。他也确实太心急了,目前我们这些读韩寒文章的庸众在这个社会还只是小众,光是这群人发展成真的大众也还需时日。\\
  但他也是意在提醒电脑前的你我,做出一点行动,发出自己的声音。
\end{itemize}

Islash

骂的痛快,有些人自己胆小,还指责别人勇敢

HomeLand Sun

以下都是个人观点:\\
笑来老师也说过他也是在``贩卖常识'',我就是庸众之一,看着笑来老师的博文有了那么一点点自己的想法(即有一点点独立思考能力,其实,这个谁都有,小狗也知道怎么做才能讨主人欢喜,不能说这不是``独立思考''吧?!),但是我感觉我不知道的我的想法要用来去实现什么?社会的,政治的,个人的目的?所以,我仍然是庸众(我甚至现在都不知道庸众怎么定义),但直觉我就是吧?!\\
我的理想,人人都是善的精英(这时候庸众就难以定义了),可能吗?笑来老师说过无所谓善恶;我们用我们的理解来回应这世界,理解是个人的,世界是客观的,怎么才能做到``物我两相忘''呢?我的途径:知道自己是庸众,改变这种状况。\\
争论和价值选择有关,是个难以语之的问题,我还是坚信李泽厚!

\begin{itemize}[<+->]
\item
  david

  李泽厚不错
\end{itemize}

HomeLand Sun

是个``度''的问题,笑来老师的这篇文章地观点超过了我所认为的(这是重点)讨论问题时的``度'',我是粉丝,因这次的观点也可能是被扔砖的粉丝,所以,我认为庸众者只会扔砖不会垒转建房者也(先自我设好防线啊,我们行动时都这样的吧?)。

pandafq

许知远的结论是:``韩寒掀起的迷狂,衬托出这个崛起大国的内在苍白、可悲、浅薄---------一个聪明的青年人说出了一些真话,他就让这个时代的神经震颤不已。与其说这是韩寒的胜利,不如说是庸众的胜利,\textbf{或是整个民族的失败。}\textbf{''据《华商报》}

stonewang

宣泄出来很畅快,便仅此而已!

kk

A:《时代》是啥杂志?权威吗?你读了几本?\\ B:不好意思,没钱订阅。\\
A:盗版的电子版你总看过吗?\\ B:也没怎么看,没时间,也就短短嘘嘘看了几年。\\
A:那你英语真好。\\ B:老子我是老美。\\ A:韩寒的文章看了吗?\\
B:韩寒是谁?\\ 。。。。\\ (和老美的真实对话)

john

看着你的文章就知道你是什么人,就知道你推崇的是个什么东西。中国人就是喜欢超越,一边大声嚷嚷让别人守法,一边又以能超越法律为荣。一边大骂别人傻逼,一边做着傻逼的事情。庸众的胜利不是指责韩寒或者韩寒的文章的,是专门影射像你这样的人,你拿起镜子照一下自己的文章,就会明白你的悲哀。但是狂热的愚昧的人往往不会自省!引用一下你自己的话。\\
``露怯了吧?没意思了吧?''

\begin{itemize}[<+->]
\item
  ⑦號桌銶

  嗯, 你承认你是one of 庸众就好\textasciitilde{}对自己认识的很全面,
  而且你一点都没露怯, 你很勇敢, 你很有意思.
  因为你蹦蹦跳跳的就出来大喊''我是庸众,
  我牛逼''\textasciitilde{}我发自肺腑滴膜拜你\ldots{}真滴\ldots{}
\end{itemize}

http://hongson hongson

呵呵 韩寒 的确是 当之无愧、\textasciitilde{}

HM\_S

感觉像是大字报时代

rbmls

\ldots{}抛开韩寒和小人\ldots{}我进来只看到漫天飞舞的屎\ldots{}

优雅

虽然不了解韩寒,但他的影响力确实令我佩服得五体投地。

Previous post:
\href{/web/20131011170823/http://wordpress.lixiaolai.com/archives/9332.html}{5月9日奇遇花园讲座}

Next post:
\href{/web/20131011170823/http://wordpress.lixiaolai.com/archives/9346.html}{America:
The Story of Us}

\href{/web/20131011170823/http://product.dangdang.com/product.aspx?product\_id=22741868}{\includegraphics{/web/20131011170823im_/http://wordpress.lixiaolai.com/wp-content/themes/thesis_182/custom/rotator/sat-essential-vocabulary-21.jpg}}

\begin{itemize}[<+->]
\item
  \subsubsection{Archives}

  Select Month September 2012 August 2012 July 2012 June 2012 May 2012
  April 2012 March 2012 February 2012 January 2012 December 2011
  November 2011 October 2011 September 2011 August 2011 July 2011 June
  2011 May 2011 April 2011 March 2011 February 2011 January 2011
  December 2010 November 2010 October 2010 September 2010 August 2010
  July 2010 June 2010 May 2010 April 2010 March 2010 February 2010
  January 2010 December 2009 November 2009 October 2009 September 2009
  August 2009 July 2009 June 2009 May 2009 April 2009 March 2009
  February 2009 January 2009 December 2008 November 2008 October 2008
  September 2008 August 2008 July 2008 June 2008 May 2008 April 2008
  March 2008 February 2008 January 2008 December 2007 November 2007
  October 2007 September 2007 August 2007 July 2007 June 2007 May 2007
  April 2007 March 2007 February 2007 January 2007 December 2006
  November 2006 October 2006 June 2006 May 2006 January 2000
\item
  \subsubsection{Categories}

  \begin{itemize}[<+->]
  \item
    \href{/web/20131011170823/http://wordpress.lixiaolai.com/archives/category/applying-stackexchange-com}{applying.stackexchange.com}
  \item
    \href{/web/20131011170823/http://wordpress.lixiaolai.com/archives/category/daily}{Daily}
  \item
    \href{/web/20131011170823/http://wordpress.lixiaolai.com/archives/category/dont-buy-bullshits}{Don't
    buy bullshits}
  \item
    \href{/web/20131011170823/http://wordpress.lixiaolai.com/archives/category/faq-sets}{FAQ问答集}
  \item
    \href{/web/20131011170823/http://wordpress.lixiaolai.com/archives/category/gmat-prep}{GMAT考试准备}
  \item
    \href{/web/20131011170823/http://wordpress.lixiaolai.com/archives/category/uncategorized}{uncategorized}
  \item
    \href{/web/20131011170823/http://wordpress.lixiaolai.com/archives/category/video-and-audio}{Video
    and Audio}
  \item
    \href{/web/20131011170823/http://wordpress.lixiaolai.com/archives/category/\%e4\%b8\%80\%e5\%88\%86\%e9\%92\%9f\%e6\%94\%b9\%e5\%8f\%98\%e7\%94\%9f\%e6\%b4\%bb-\%e6\%88\%91\%e6\%98\%af\%e8\%b0\%81\%ef\%bc\%9f-in-chinese}{一分钟改变生活------我是谁?}
  \item
    \href{/web/20131011170823/http://wordpress.lixiaolai.com/archives/category/books-that-should-not-be-missed}{书------不得不看}
  \item
    \href{/web/20131011170823/http://wordpress.lixiaolai.com/archives/category/people}{人物}
  \item
    \href{/web/20131011170823/http://wordpress.lixiaolai.com/archives/category/readings-for-writing}{作文素材}
  \item
    \href{/web/20131011170823/http://wordpress.lixiaolai.com/archives/category/clear-writing}{写清楚(系列)}
  \item
    \href{/web/20131011170823/http://wordpress.lixiaolai.com/archives/category/learning-learning-to-learn-better}{学习学习再学习}
  \item
    \href{/web/20131011170823/http://wordpress.lixiaolai.com/archives/category/versatile}{强人高手}
  \item
    \href{/web/20131011170823/http://wordpress.lixiaolai.com/archives/category/thinking-tools}{思维工具}
  \item
    \href{/web/20131011170823/http://wordpress.lixiaolai.com/archives/category/thinking-clearly-series}{想明白(系列)}
  \item
    \href{/web/20131011170823/http://wordpress.lixiaolai.com/archives/category/response-to-readers}{我要问笑来}
  \item
    \href{/web/20131011170823/http://wordpress.lixiaolai.com/archives/category/finding-examples}{找例子玩}
  \item
    ►\href{/web/20131011170823/http://wordpress.lixiaolai.com/archives/category/time-as-a-friend}{把时间当作朋友}

    \begin{itemize}[<+->]
    \item
      \href{/web/20131011170823/http://wordpress.lixiaolai.com/archives/category/time-as-a-friend/time-as-a-friend-3rd-edition}{《把时间当作朋友》第三版}
    \end{itemize}
  \item
    \href{/web/20131011170823/http://wordpress.lixiaolai.com/archives/category/public-speaking-skills-for-everyone}{普通人的当众讲话技能}
  \item
    \href{/web/20131011170823/http://wordpress.lixiaolai.com/archives/category/things-better-learned-before-getting-a-job}{求职前最好学会的那些事儿}
  \item
    \href{/web/20131011170823/http://wordpress.lixiaolai.com/archives/category/qa-for-application}{留学问答}
  \item
    \href{/web/20131011170823/http://wordpress.lixiaolai.com/archives/category/movie-time}{看看电影}
  \item
    \href{/web/20131011170823/http://wordpress.lixiaolai.com/archives/category/marvelous-brain}{神奇的大脑}
  \item
    \href{/web/20131011170823/http://wordpress.lixiaolai.com/archives/category/temp-list}{算是流水帐}
  \item
    \href{/web/20131011170823/http://wordpress.lixiaolai.com/archives/category/diabetes-common-knowledge}{糖尿病常识}
  \item
    \href{/web/20131011170823/http://wordpress.lixiaolai.com/archives/category/us-universities-and-colleges}{美国大学}
  \item
    \href{/web/20131011170823/http://wordpress.lixiaolai.com/archives/category/\%e7\%be\%8e\%e5\%9b\%bd\%e5\%a4\%a7\%e5\%ad\%a6\%e6\%9c\%ac\%e7\%a7\%91\%e7\%94\%b3\%e8\%af\%b7diy\%e7\%ae\%80\%e6\%98\%8e\%e6\%8c\%87\%e5\%8d\%97}{美国大学本科申请DIY简明指南}
  \item
    ►\href{/web/20131011170823/http://wordpress.lixiaolai.com/archives/category/english-learning}{英语相关文章}

    \begin{itemize}[<+->]
    \item
      \href{/web/20131011170823/http://wordpress.lixiaolai.com/archives/category/english-learning/1800-hours}{1800小时
      -- 高中两年慢跑冲进美国名校}
    \item
      \href{/web/20131011170823/http://wordpress.lixiaolai.com/archives/category/english-learning/awa-basic-instruction}{AWA(GRE/GMAT写作)考试简明教程}
    \item
      \href{/web/20131011170823/http://wordpress.lixiaolai.com/archives/category/english-learning/toefl-ibt-qualified-samples-for-ets-official-185-topic-pool}{TOEFL
      iBT 高分作文}
    \item
      \href{/web/20131011170823/http://wordpress.lixiaolai.com/archives/category/english-learning/master-toefl-essential-vocabulary-in-21-days}{TOEFL核心词汇21天突破}
    \item
      \href{/web/20131011170823/http://wordpress.lixiaolai.com/archives/category/english-learning/everyone-can-use-english}{人人都能用英语}
    \item
      \href{/web/20131011170823/http://wordpress.lixiaolai.com/archives/category/english-learning/how-to-accelerate-reading-speed}{如何提高阅读速度}
    \item
      \href{/web/20131011170823/http://wordpress.lixiaolai.com/archives/category/english-learning/toefl-listening-and-speaking}{托福听说训练}
    \item
      \href{/web/20131011170823/http://wordpress.lixiaolai.com/archives/category/english-learning/toefl-ibt-topic-vocabulary}{新托福iBT词汇分类突破}
    \item
      \href{/web/20131011170823/http://wordpress.lixiaolai.com/archives/category/english-learning/toefl-vocabulary-classification}{新托福词汇分类突破}
    \item
      \href{/web/20131011170823/http://wordpress.lixiaolai.com/archives/category/english-learning/test-preparation}{留学考试准备}
    \item
      \href{/web/20131011170823/http://wordpress.lixiaolai.com/archives/category/english-learning/earn-vocabulary-with-me}{跟我攒单词}
    \end{itemize}
  \item
    \href{/web/20131011170823/http://wordpress.lixiaolai.com/archives/category/opinions}{言论与八卦}
  \item
    ►\href{/web/20131011170823/http://wordpress.lixiaolai.com/archives/category/computer-related}{计算机相关}

    \begin{itemize}[<+->]
    \item
      ►\href{/web/20131011170823/http://wordpress.lixiaolai.com/archives/category/computer-related/all-about-apple}{All
      about Apple}

      \begin{itemize}[<+->]
      \item
        \href{/web/20131011170823/http://wordpress.lixiaolai.com/archives/category/computer-related/all-about-apple/ios}{iOS}
      \item
        \href{/web/20131011170823/http://wordpress.lixiaolai.com/archives/category/computer-related/all-about-apple/iphone}{iPhone}
      \item
        \href{/web/20131011170823/http://wordpress.lixiaolai.com/archives/category/computer-related/all-about-apple/freeware}{免费软件}
      \item
        \href{/web/20131011170823/http://wordpress.lixiaolai.com/archives/category/computer-related/all-about-apple/basics}{入门}
      \item
        \href{/web/20131011170823/http://wordpress.lixiaolai.com/archives/category/computer-related/all-about-apple/shareware}{共享软件}
      \item
        \href{/web/20131011170823/http://wordpress.lixiaolai.com/archives/category/computer-related/all-about-apple/tips}{小技巧}
      \end{itemize}
    \item
      \href{/web/20131011170823/http://wordpress.lixiaolai.com/archives/category/computer-related/auto-it}{Auto-It}
    \item
      \href{/web/20131011170823/http://wordpress.lixiaolai.com/archives/category/computer-related/firefox}{Firefox}
    \item
      \href{/web/20131011170823/http://wordpress.lixiaolai.com/archives/category/computer-related/google}{Google}
    \item
      \href{/web/20131011170823/http://wordpress.lixiaolai.com/archives/category/computer-related/rails-notes}{rails
      notes}
    \item
      \href{/web/20131011170823/http://wordpress.lixiaolai.com/archives/category/computer-related/ubuntu}{Ubuntu}
    \item
      \href{/web/20131011170823/http://wordpress.lixiaolai.com/archives/category/computer-related/wordpress-computer-related}{WordPress}
    \item
      \href{/web/20131011170823/http://wordpress.lixiaolai.com/archives/category/computer-related/blogging-experience}{博客经验谈}
    \item
      \href{/web/20131011170823/http://wordpress.lixiaolai.com/archives/category/computer-related/computer-and-internet-common-sense-for-dummies}{网络常识菜鸟笔记}
    \end{itemize}
  \item
    \href{/web/20131011170823/http://wordpress.lixiaolai.com/archives/category/lecture-schedule}{讲座行程}
  \item
    \href{/web/20131011170823/http://wordpress.lixiaolai.com/archives/category/resouces-recommended}{资源推荐}
  \item
    \href{/web/20131011170823/http://wordpress.lixiaolai.com/archives/category/isssue-and-argumentation}{问题与争论}
  \item
    \href{/web/20131011170823/http://wordpress.lixiaolai.com/archives/category/scratches}{随便想想}
  \item
    \href{/web/20131011170823/http://wordpress.lixiaolai.com/archives/category/\%e9\%9a\%8f\%e4\%be\%bf\%e6\%83\%b3\%e6\%83\%b3}{随便想想}
  \item
  \end{itemize}
\end{itemize}

© 2011 lixiaolai.com -- All rights reserved. -- Tell whom you should
tell.
